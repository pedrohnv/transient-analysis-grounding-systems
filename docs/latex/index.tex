High Performance implementation of the Hybrid Electromagnetic Model and its variants.

The documentation of the code can be read at {\texttt{ https\+://pedrohnv.\+github.\+io/hp\+\_\+hem/}}

The Hybrid Electromagnetic Model can be used to simulate any situation where the thin wire hypothesis is valid. See [1] for the mathematical formulation.

[1] V\+I\+S\+A\+C\+RO, S.; S\+O\+A\+R\+ES, A. H\+EM\+: A model for simulation of lightning-\/related engineering problems. I\+E\+EE Transactions on power delivery, v. 20, n. 2, p. 1206-\/1208, 2005.

Cite the current release\+: {\texttt{ }}

The permanent D\+OI is {\texttt{ https\+://doi.\+org/10.\+5281/zenodo.\+2644010,}} which always resolves to the last release.

This library is intended to be easy to use. Interfaces to use it from other programming languages are provided for (work in progress, Julia is the recommended one)\+:


\begin{DoxyItemize}
\item Julia {\texttt{ https\+://github.\+com/pedrohnv/hp\+\_\+hem\+\_\+julia}}
\item Matlab {\texttt{ https\+://github.\+com/pedrohnv/hp\+\_\+hem\+\_\+matlab}}
\end{DoxyItemize}

The {\ttfamily examples} folder contains various C files which reproduce results published in the technical literature. Use them as starting point to build your own cases in pure C if you want maximum performance. Examples of use from other programming languages are in their respective repository.

Dependencies\+:
\begin{DoxyItemize}
\item {\texttt{ Cubature}}
\item {\texttt{ Open\+B\+L\+AS}}
\item {\texttt{ L\+A\+P\+A\+CK}}
\item {\texttt{ F\+F\+T\+W3}} 
\end{DoxyItemize}